\section*{Anexos}
\addcontentsline{toc}{section}{Anexos}

\subsection*{Dicionário WBS}
\addcontentsline{toc}{subsection}{Dicionário WBS}

Relativamente ao \textbf{WBS} idealizado para este projeto e ao nível de \textit{4.1.1 | Avaliação do nível de cumprimento do \textbf{RGPD} por parte da \textbf{UPI}} (trabalho de 4\mycirc nível), este irá contar com vários subscomponentes, tais como:

\begin{itemize}
	\item Princípios relativos ao tratamento/qualidade dos dados;
	\item Consentimento;
	\item Categorias especiais de dados;
	\item Informação;
	\item Confidencialidade;
	\item Direitos de Acesso;
	\item Retificação, Cancelamento, Oposição, Eliminação, Limitação do tratamento e Portabilidade dos dados;
	\item Comunidação dos dados;
	\item Acesso a dados por conta de terceiros;
	\item Registo de atividades de tratamento;
	\item Transferências internacionais;
	\item Notificação de violações de segurança;
	\item Avaliação de impacto da privacidade;
	\item Segurança do tratamento;
	\item Modelo de governança.
\end{itemize}

No que toca ao ponto \textit{5.1.1. | Definição de estratégia de adequação ao RGPD por parte da \textbf{UPI}}, (trabalho de 4\mycirc nível), este irá contar com:

\begin{itemize}
	\item O nível de maturidade da \textbf{UPI} eos casos específicos identificados, bem como o modelo de implementação mais adequeado;
	\item A planificação mais adequada, ex., com um critério de baixo risco e impacto;
	\item Nível de maturidade que será possível atingir no cumprimento dos processos estabelecidos pelo \textbf{RGPD},atendendo aos diferentes recursos e elementor disponíveis atualmente;
	\item Limitaçõese impedimentos atuais existentes na\textbf{UPI} para implementar os diferentes processos \textbf{RGPD};
	\item Momento em que será oportuno introduzir elementos adicionais (capacidades, processos e tecnologia),de forma aincrementar o nível de maturidade no cumprimento dos diferentes processos do \textbf{RGPD}.
\end{itemize}