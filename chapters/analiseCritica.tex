\section{Análise crítica}
Neste capítulo, irá ser efetuada uma análise crítica referente à adequação dos processos e áreas de conhecimento do PMBOK à gestão do projeto e eventuais propostas de adaptação.

\subsection{Processos}
A PMBOK define 42 processos e distribui-os por cinco principais grupos:
\begin{itemize}
    \item Inicio;
    \item Planeamento;
    \item Execução;
    \item Monitoramento e controlo;
    \item Finalização.
\end{itemize}

\subsection{Áreas de conhecimento}
Todos estes processos referidos previamente podem ser separados por áreas de conhecimento, sendo elas:
\begin{itemize}
    \item Gestão da integração do projeto;
    \item Gestão do \textit{scope} do projeto;
    \item Gestão do tempo de projeto;
    \item Gestão dos custos do projeto;
    \item Gestão da qualidade do projeto;
    \item Gestão dos recursos humanos do projeto;
    \item Gestão das comunicações do projeto;
    \item Gestão dos riscos do projeto;
    \item Gestão dos riscos do projeto.
\end{itemize}

Analisando todos estes processos, podemos encontrar alguns que se destacam e que de certa forma causariam um impacto no nosso projeto.

\clearpage

\subsection{Vantagens}

\begin{itemize}
    \item Desenvolver o plano de gestão do projeto é um dos processos que possui uma maior importância, pois a sua criação irá auxiliar toda a execução do projeto servindo como um guia.
    \item Criar uma estrutura analítica do projeto é igualmente relevante e que neste projeto, este processo foi adotado, através da criação do WBS.
    \item Estimar as durações das atividades de forma a gerir melhor o tempo necessário para cada fase do projeto.
    \item  O cronograma foi desenvolvido com o propósito de determinar o caminho crítico do projeto e também definir a duração e o rumo que o projeto irá tomar. 
    \item Orientar e gerir a execução do projeto conseguindo assim assegurar que o projeto está a seguir o rumo indicado.
    \item Controlar os custos de modo a manter os custos estáveis e evitar subidas acentuadas do mesmo.
\end{itemize}

Estas são apenas algumas das medidas mais importantes, usadas para a realização deste trabalho e as respetivas vantagens oferecidas pelas mesmas.