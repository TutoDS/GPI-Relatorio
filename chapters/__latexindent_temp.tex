\section{Pressupostos Baseline 1}

Os pressupostos usados para a \emph{Baseline 1} serão tudo aquilo que é usado como base para a criação da mesma. Dito isto, temos então três pilares essenciais sendo eles o âmbito, a equipa e a duração do projeto.

\subsection{Âmbito}
Além do objetivo principal temos outros que são subentendidos ou adjacentes ao mesmo, sendo estes igualmente importantes e que visam explorar o âmbito do produto de forma mais completa: 

\begin{itemize}
    \item Identificação do impacto do RGPD sobre os processos organizacionais e Sistemas de Informação da UPI;
    \item Identificação do nível de cumprimentos dos diferentes elementos afetados, com respeito aos requisitos e exigências estabelecidos pelo RGPD;
    \item Identificação dos riscos a que se encontram expostos os tratamentos de dados relacionados com os diferentes processos organizacionais em que se decompõe a atividade da UPI, considerando os aspetos setoriais específicos da sua missão e atribuições;
    \item Definição de orientações, transversais a toda a UPI, para o cumprimento dos novos requisitos introduzidas pelo RGPD, tendo em consideração os diferentes fatores;
    \item Estabelecer os planos para o cumprimento dos requisitos do RGPD nas diferentes Unidades Orgânicas da UPI.
\end{itemize}
  

Este âmbito da avaliação de impacto de privacidade engloba todas as Unidades Orgânicas e Serviços da UPI e para facilitar este trabalho recorremos à construção do \glsxtrshort{wbs}, realizado no trabalho prático anterior 





\addCenteredImg{wbs.png}{WBS realizado para o projeto em questão}